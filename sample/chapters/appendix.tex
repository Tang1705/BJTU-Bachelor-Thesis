\begin{appendix}

\vspace{5bp}
\noindent{\zihao{3} \heiti 附录A 外文文献及翻译}

\vspace{25bp}
\noindent{\zihao{3} \heiti 外文文献}

\vspace{25bp}
\centering{\zihao{-3}Learning Scene Structure Guidance via Cross-Task Knowledge Transfer for Single Depth Super-Resolution}

\vspace{20bp}
\centering{\zihao{5}Baoli Sun$^1$, Xinchen Ye$^{1,2*}$, Baopu Li$^3$, Haojie Li$^{1,2}$, Zhihui Wang$^{1,2}$, Rui Xu$^{1,2}$

$^1$International School of Information Science \& Engineering, Dalian University of Technology, China\newline
$^2$Key Laboratory for Ubiquitous Network and Service Software of Liaoning Province, China\newline
$^3$Baidu Research, USA
}

\vspace{25bp}
\noindent \justifying{\bfseries{Abstract.}} Existing color-guided depth super-resolution (DSR) approaches require paired RGB-D data as training samples where the RGB image is used as structural guidance to recover the degraded depth map due to their geometrical similarity. However, the paired data may be limited or expensive to be collected in actual testing environment. Therefore, we explore for the first time to learn the cross-modality knowledge at training stage, where both RGB and depth modalities are available, but test on the target dataset, where only single depth modality exists. Our key idea is to distill the knowledge of scene structural guidance from RGB modality to the single DSR task without changing its network architecture. Specifically, we construct an auxiliary depth estimation (DE) task that takes an RGB image as input to estimate a depth map, and train both DSR task and DE task collaboratively to boost the performance of DSR. Upon this, a cross-task interaction module is proposed to realize bilateral cross-task knowledge transfer. First, we design a cross-task distillation scheme that encourages DSR and DE networks to learn from each other in a teacher-student role-exchanging fashion. Then, we advance a structure prediction (SP) task that provides extra structure regularization to help both DSR and DE networks learn more informative structure representations for depth recovery. Extensive experiments demonstrate that our scheme achieves superior performance in comparison with other DSR methods. Our code available at: https://github.com/Sunbaoli/dsr-distillation.






\end{appendix}