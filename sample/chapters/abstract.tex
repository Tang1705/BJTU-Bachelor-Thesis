\begin{chineseabstract}

	\noindent\bjtusongti{\textbf{摘要:}}深度图像超分辨率重建是工业中具有较高实际应用需求的任务。现有的颜色引导 的深度图像超分辨率重建方法通常需要额外的分支来从彩色图像中提取高频信息,用于 指导低分辨率深度图像的重建。但是,由于彩色图像和深度图像两种模态之间仍然存在 着一些差异,因此在特征维度或边缘图维度上进行直接的信息传输无法获得令人满意的 结果,甚至可能在 RGB-D 图像对之间结构不一致的区域造成纹理复制等问题。
	
受多任务学习的启发,本文提出了联合单目深度估计的深度图像超分辨率重建网络。 由于两个任务可以共享训练数据,因而在将两个任务统一于一个联合学习网络时,无需 引入其他监督标签。对于两个子网的交互,本文采用了差异化的指导策略,并相应地设 计了两个桥接器。一个是在特征编码过程中的高频注意力桥,它被设计用于从单目深度 估计任务中学习高频信息以指导深度图像超分辨率重建任务。另一个是在深度图像重建 过程中的内容引导桥,它被设计用于为单目深度估计任务提供从深度图像超分辨率重建 任务中学到的内容指导。整个网络体系结构具有高度的可移植性,可以为关联深度图像 超分辨率重建任务和单目深度估计任务提供范例。在公开的基准数据集上进行的实验表 明,本文提出的方法取得了具有竞争优势的性能。

高质量和高分辨率的深度图像在自动驾驶、三维重建、人机交互和虚拟现实等诸多 领域至关重要,因而更好地从低分辨率的深度图像恢复出高分辨率的深度图像将有助于 推动下游任务的发展和实际应用。
	\newline		
	\newline
	\noindent \bjtusongti {\textbf{关键词:}深度图像;超分辨率重建;单目深度估计;多任务学习}
\end{chineseabstract}